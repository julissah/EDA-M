%package list
\documentclass{article}
\usepackage[top=3cm, bottom=3cm, outer=3cm, inner=3cm]{geometry}
\usepackage{graphicx}
\usepackage{url}
%\usepackage{cite}
\usepackage{hyperref}
\usepackage{array}
\usepackage{multicol}
\newcolumntype{x}[1]{>{\centering\arraybackslash\hspace{0pt}}p{#1}}
\usepackage{natbib}
\usepackage{pdfpages}
\usepackage{multirow}
\usepackage{float}
\usepackage[normalem]{ulem}
\useunder{\uline}{\ul}{}


%%%%%%%%%%%%%%%%%%%%%%%%%%%%%%%%%%%%%%%%%%%%%%%%%%%%%%%%%%%%%%%%%%%%%%%%%%%%
%%%%%%%%%%%%%%%%%%%%%%%%%%%%%%%%%%%%%%%%%%%%%%%%%%%%%%%%%%%%%%%%%%%%%%%%%%%%
\newcommand{\csemail}{vmachacaa@unsa.edu.pe}
\newcommand{\csdocente}{Vicente Machaca Arceda}
\newcommand{\cscurso}{Estructura de Datos y Algoritmos}
\newcommand{\csuniversidad}{Universidad Nacional de San Agustín}
\newcommand{\csescuela}{Maestría en Ciencia de la Computación}
\newcommand{\cspracnr}{01}
\newcommand{\cstema}{--}
%%%%%%%%%%%%%%%%%%%%%%%%%%%%%%%%%%%%%%%%%%%%%%%%%%%%%%%%%%%%%%%%%%%%%%%%%%%%
%%%%%%%%%%%%%%%%%%%%%%%%%%%%%%%%%%%%%%%%%%%%%%%%%%%%%%%%%%%%%%%%%%%%%%%%%%%%


\usepackage[english,spanish]{babel}
\usepackage[utf8]{inputenc}
\AtBeginDocument{\selectlanguage{spanish}}
\renewcommand{\figurename}{Figura}
\renewcommand{\refname}{Referencias}
\renewcommand{\tablename}{Tabla} %esto no funciona cuando se usa babel
\AtBeginDocument{%
	\renewcommand\tablename{Tabla}
}

\usepackage{fancyhdr}
\pagestyle{fancy}
\fancyhf{}
\setlength{\headheight}{30pt}
\renewcommand{\headrulewidth}{1pt}
\renewcommand{\footrulewidth}{1pt}
\fancyhead[L]{\raisebox{-0.2\height}{\includegraphics[width=3cm]{logo_unsa}}}

\fancyhead[C]{}
\fancyhead[R]{\fontsize{7}{7}\selectfont	\csuniversidad \\ \csescuela \\ \textbf{\cscurso} }
\fancyfoot[L]{MSc. Vicente Machaca}
\fancyfoot[C]{\cscurso}
\fancyfoot[R]{Página \thepage}







\begin{document}
	
	\vspace*{10px}
	
	\begin{center}	
		\fontsize{17}{17} \textbf{ Práctica \cspracnr}
	\end{center}
	%\centerline{\textbf{\underline{\Large Título: Informe de revisión del estado del arte}}}
	%\vspace*{0.5cm}
	

	\begin{table}[h]
		\begin{tabular}{|x{4.7cm}|x{4.8cm}|x{4.8cm}|}
			\hline 
			\textbf{DOCENTE} & \textbf{CARRERA}  & \textbf{CURSO}   \\
			\hline 
			\csdocente & \csescuela & \cscurso    \\
			\hline 
		\end{tabular}
	\end{table}	
	
	
	\begin{table}[h]
		\begin{tabular}{|x{4.7cm}|x{4.8cm}|x{4.8cm}|}
			\hline 
			\textbf{PRÁCTICA} & \textbf{TEMA}  & \textbf{DURACIÓN}   \\
			\hline 
			\cspracnr & \cstema & 3 horas   \\
			\hline 
		\end{tabular}
	\end{table}
	
	
	\section{Datos de los estudiantes}
	\begin{itemize}
		\item Grupo: 3
		\item Integrantes: 
		\begin{itemize}
			\item Lizarraga Mendoza David Jesus
			\item Saenz Mamani Alex Alberto
			\item Huaman Hilari Julissa Zaida
			\item Chara Condori Julio Cesar
			\item Acuña Chavez Melvin
		\end{itemize}		
	\end{itemize}
	
	
	

	
	\section{Ejercicios}\label{sec:ejercicios}
	\begin{enumerate}
		\item Preparación de los datos. Usted debe generar varios conjuntos de datos (archivos .txt ), por
ejemplo va a generar números aleatorios y los va a almacenar en varios archivos. Cada archivo
deberá contener respectivamente 100, 500, 1000, 2000, 3000, ... , 10000, 20000, 30000, ... ,100000
datos.\\
		
		Solución \\
		Se generó 21 archivos txt utilizando el siguiente algoritmo desarrollado en python, ubicado en el siguiente link: \url{https://github.com/julissah/EDA-M/blob/development/GenarateFile/generatefile.py}
		
		\item Implemente los siguientes algoritmos en C++ y Python:\\
		
		Solución  \\
		Se implentó los siguientes algoritmos de ordenamiento, los cuales estan ubicados en el siguiente link \url{https://github.com/julissah/EDA-M/tree/development}:\\\\
		Bubble sort\\
        Counting sort\\
        Heap sort\\
        Insertion sort\\
        Merge sort\\
        Quick sort\\
        Selection sort\\
        
        \item Realice comparaciones del tiempo de procesamiento de cada algoritmo por cada lenguaje de
programación. Debe generar las siguientes gráficas:
		
	\end{enumerate}


	
	%\clearpage
	%\bibliographystyle{apalike}
	%\bibliographystyle{IEEEtranN}
	%\bibliography{bibliography}
		
	
\end{document}